\documentclass[9pt]{article}

\usepackage{fullpage}
\usepackage{hyperref}
\usepackage{enumitem}
\usepackage{multicol}
% \usepackage[normalem]{ulem}

\addtolength{\topmargin}{-.25in}
\addtolength{\textheight}{0.5in}
% \setlength{\parindent}{0pt}
\setlength{\multicolsep}{2pt}

\title{Syllabus: ASEN 3728 Aircraft Dynamics}
\author{Professor Zachary Sunberg}
\date{Spring 2024}

\begin{document}

\maketitle

\section*{Meetings}

T/TH 8:30-9:45, AERO 120 -- Lecture video will automatically be posted online - see Piazza for link.

\section*{Course Staff}

\subsection*{Instructor}
Professor Zachary Sunberg\\
AERO 263 \href{mailto://zachary.sunberg@colorado.edu}{\nolinkurl{zachary.sunberg@colorado.edu}} (please make a public or private Piazza post before emailing)\\
\textbf{Office Hours}: Posted on Piazza

\subsection*{Teaching Assistants}

\textbf{Office Hours}: Posted on Piazza

\begin{itemize}[noitemsep]
\item Scott Mckinley 
  \href{mailto:scott.mckinley@colorado.edu}{\nolinkurl{scott.mckinley@colorado.edu}}
\item Ben Kraske
  \href{mailto:benjamin.kraske@colorado.edu}{\nolinkurl{benjamin.kraske@colorado.edu}}
\item Yusif Razzaq
  \href{mailto:Yusif.Razzaq@colorado.edu}{\nolinkurl{Yusif.Razzaq@colorado.edu}}
\item Vaibhavi Thakur
  \href{mailto:Vaibhavi.Thakur@colorado.edu}{\nolinkurl{Vaibhavi.Thakur@colorado.edu}}
\item Paraksh Vankawala
  \href{mailto:paraksh.vankawala@colorado.edu}{\nolinkurl{paraksh.vankawala@colorado.edu}}
\end{itemize}

\section*{Textbook}

\textbf{Bernard Etkin and Lloyd Reid, \textit{Dynamics of Flight: Stability and Control, 3rd Edition}}. 1996, John Wiley and Sons.

\section*{Prerequisites}

ASEN 2002, 2003, 2004, and APPM 2360 (min grade C-).

\section*{Overview}

This course covers the key ideas that enable: (i) an understanding of
how aircraft work and tools for quantitative analysis, and (ii) design
methods to achieve specified dynamical behavior. Because aircraft exist
in many different forms, and new designs continue to be developed, the
focus is on the common principles that underlie atmospheric flight, so
that a solid basis can be formed for future work in any direction.
Concrete treatment of these ideas, tools, and methods is provided
through working problems consisting
of analysis, simulation, and design, including development of
simulation models for two very different vehicles: a quad-copter and a
conventional airplane.

In their full expression, aircraft dynamics possess astounding
complexity. It is a tribute to the ideas developed by aviation's
pioneers that a relatively simple understanding can often be obtained,
leading to clear insights and design principles. While these concepts
are not inherently difficult, they do lie outside most common
experience, and they depend on new nomenclature and strange notation
that can seem overwhelming at first. It is only through diligent and
careful use of this new language that the underlying simplicity can be
grasped and conveyed on exams; mastery of the language of aircraft
dynamics is perhaps the most important predictor for success in the
course.

The course has been designed to develop a conceptual grasp of the key
ideas below, and to demonstrate proficiency in using these concepts to
solve problems, construct and validate simulations, and to explain
behaviors and results obtained. In particular, engineering reasoning
skills using these concepts are stressed in assignment solutions and
examinations. The key learning objectives are:\\

\begin{itemize}[nosep]
\item Vector mechanics
  \begin{itemize}[nosep]
  \item Vector representation in coordinate frames
  \item Change of coordinate frame representation (coordinate rotation)
  \item Relative motion, frame derivatives
  \item Change of derivative frame: velocity rule
  \end{itemize}
\item How aircraft dynamics models are created and what the terms mean
  \begin{itemize}[nosep]
  \item 3D rigid body translational model
    \begin{itemize}[nosep]
    \item Kinematics
    \item Dynamics, external forces
    \item Effects of wind
    \end{itemize}
  \item 3D rigid body rotational model
      \begin{itemize}[nosep]
    \item Kinematics, Euler angle attitude representation
    \item Dynamics, Euler moment equations, external moments
    \end{itemize}
  \item External forces and moments
      \begin{itemize}[nosep]
    \item Aerodynamic effects
    \item Control effects
    \item Steady flight conditions, trim states
    \end{itemize}
  \end{itemize}
\item How aircraft dynamics models are simulated
    \begin{itemize}[nosep]
    \item State space models
    \item Numerical integration
  \end{itemize}
\item How dynamical behavior is understood and specified
    \begin{itemize}[nosep]
    \item Linearization
    \item Decoupling
    \item Stability derivatives
    \item Modal solutions
    \item Stability characterizations
    \item Modal specifications
  \end{itemize}
\item How feedback control is designed to meet behavioral objectives
    \begin{itemize}[nosep]
    \item Sensor/feedback selection, control structure and gain selection
    \item Effects on mode eigenvalues
  \end{itemize}
\end{itemize}

\section*{Course Components}\label{course-components}

Material and concepts are introduced, and student mastery is evaluated
using several mechanisms throughout the course:

\textbf{Reading} -- The textbook provides the essential basis for the
course, including the concepts, terminology, notation, methods, and
examples used to convey the course topics. Specific reading assignments
will be given covering key sections of the book; some book sections are
not covered in the course. Some supplementary material will also be
provided. {The textbook contains a wealth of information, but the
concepts and notation are new to most; some sections need to be read
more than once to fully grasp the material}\emph{.}

\textbf{Lectures} -- These are intended to emphasize key ideas and
methods that make the material easier to grasp. They are therefore a
counterpart to the reading, not a replacement. {The value of lectures is
dependent on your participation in them}. Passive ``watching'' will
provide little benefit. {Active note taking is critical to developing
first-hand familiarity with the notation, terminology, and methods, and
to gaining comfort in using them}. Although lectures will be recorded,
this is a poor substitute for your own lecture notes. Questions are
encouraged during lectures and will be prompted often.

\textbf{Homework} -- In the instructor's opinion, homework is the most important tool for learning in this class because it provides essential individual practice in preparation for the exams and experience solving problems that will be faced in the student's career. Homework will consist of
solving problems of varying difficulty and sometimes will also involve
computing. Collaboration on homework is encouraged, but the work you submit must be your own. Students are encouraged to use homework as a means to ensure
their individual mastery of the subject.

\textbf{In-Class and Reading Quizzes} -- These will cover the reading material, and
lectures.  They will usually consist of true-false and multiple-choice-style questions.

\textbf{Exams} -- These are the primary means of evaluation of your
individual grasp of the course material. Exams will include both conceptual questions and
quantitative problems. Precise use of terminology and notation is
stressed. The final exam is comprehensive in that it will contain
material from the entire course, but emphasis will be placed on the
final portion of the course material. \textbf{There will
be a statute of limitations on when exam grades can be corrected. Any
corrections on exam scores must be made before the next exam, or two
weeks after the exam was returned, whichever comes later}. The only
corrections made after this time period will be for simple addition
errors in scoring.

\section*{Websites}

\begin{itemize}[nosep]
    \item \href{https://piazza.com}{\textbf{Piazza}} will host course discussions, announcements, and host solutions that are not posted publicly. Students should ask questions here rather than emailing the course staff unless there is an important reason. The class signup link is at \url{https://piazza.com/colorado/spring2024/asen3728}.
    \item \textbf{Gradescope} will be used for assignments.
    \item \href{https://github.com/zsunberg/Aircraft-Dynamics-Materials}{\textbf{Github}} will be used to host all course materials. You can download all materials without using git, but learning the basics may be much more convenient, not to mention useful for your future career.
    \item \textbf{Canvas} will be used minimally as a landing page and anything that cannot be accommodated by the above websites.
\end{itemize}

\section*{Attendance and Participation}

Learning is a collaborative effort between the instructor and students. Students are expected to attend all lectures, ask questions, and participate in discussions. The course staff will encourage attendance through in-class quizzes. \textbf{If a student needs to miss class occasionally, please do NOT notify the course staff.} Several of the lowest in-class quiz scores will be dropped to accommodate absences (see grading breakdown below).

\section*{Grading Philosophy}

Grades are assigned according to an absolute standard designed to
indicate your level of competence in the course material. The final
grade indicates your readiness to continue to the next level in the
curriculum. {The AES faculty have set these standards based on our
education, experience, interactions with industry, government
laboratories, others in academy, and according to the criteria
established by the ABET accreditation board}.

The course grade is primarily dependent on individual measures of
competency, i.e., exams. The other course assignments are designed to
enrich the learning experience and to enhance individual performance,
not to substitute for sub-standard individual competency. This policy
makes it important to use the assignments to enhance your learning.

Grades for the course are earned set based on the following criteria:
\begin{quote}
A, A- Demonstrates mastery of the course material in both conceptual and
quantitative aspects.

B+, B Demonstrates comprehensive understanding of the material, with a
solid conceptual grasp of key concepts and strong quantitative work.

B-, C+ Demonstrates good understanding of most key concepts, with few
major quantitative errors.

C Demonstrates satisfying understanding of the material with sufficient
quantitative work.

C- Demonstrates adequate understanding of the material to proceed to the
next level; sufficient quantitative work.

D Very little understanding is evident, consistently poor quantitative
work.

F Unsatisfactory performance.
\end{quote}

Graders will assess whether responses provided by
students reflect knowledge, understanding and reasoning processes that
{meaningfully contribute} to answering questions posed on assignments.
Vacuous responses, e.g., repeating questions,
listing buzzwords, irrelevant diagram drawing, etc., will not suffice. This subject is difficult and
non-intuitive, and since this is the first time most (if not all)
students have seen this material, it is naturally assumed that all
students must work hard and put in effort to learn the concepts.
Therefore, hard work is necessary, but not sufficient by itself, to do
well. Your effort must translate to demonstrable individual
understanding for success.

\subsection*{Grade Breakdown}

\begin{itemize}[noitemsep]
    \item \textbf{5\% In-class Quizzes.} (lowest 4 dropped)
    \item \textbf{5\% Reading Quizzes.} (lowest 2 dropped)
    \item \textbf{16\% Homework.} (lowest dropped)
    \item \textbf{44\% Two Midterm Exams.} (22\% each)
    \item \textbf{30\% Final Exam.}
\end{itemize}

\subsection*{Late Policy}

To ensure proper progression through the course, students are expected to begin assignments early and submit homework assignments on time. However, in order to provide for minor unforeseen events or responsibilities, students may turn in late homework assignments within 72 hours of the due date with a 10\% penalty. No homework will be accepted after 72 hours beyond the due date. The lowest homework score will be dropped to account for missed assignments.

\section*{Additional Policies}

{\small
    \input{required.tex}
}
\end{document}
