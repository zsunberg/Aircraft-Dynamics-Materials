\documentclass{article}

\usepackage{fullpage,amsmath,amsthm,graphicx,enumitem}
\usepackage{hyperref}
\usepackage{amssymb}
\usepackage{wasysym}
\usepackage{empheq}

\theoremstyle{definition}
\newtheorem{question}{Question}

\newcommand{\option}{{\Large$\Square$ }}
\def\IfSoln#1{\IfFileExists{solutions/#1}{\input{solutions/#1}}{}}

\title{ASEN 3728 Aircraft Dynamics\\Written Homework 3}

\date{Due date listed on Gradescope.}

\begin{document}

\maketitle

\begin{question}

Consider a quadrotor with $I_y=3\ \text{kg m}^2$. Perform a closed-loop modal analysis of the quadrotor's longitudinal $\Delta \theta$ and $\Delta q$ motion using the information below. Subscripts on the natural frequencies and damping ratios correspond to the eigenvalues of the state space model for the motion.
\begin{align*}
    \Delta M_c &= -k_1 \Delta q - k_2 \Delta \theta \\
    \omega_{n,1,2} &= 1.8 \quad \text{[rad/s]} \\
    \zeta_{1,2} &= 0.5
\end{align*}

\begin{enumerate}
    \item Calculate the values of $k_1$ and $k_2$, as well as the eigenvalues $\lambda_1$ and $\lambda_2$ of the size $2\times2$ state space model $\mathbf{A}$ matrix.
    \item At $t=0$, the quadrotor's state is $\mathbf{x}(0) = 5\mathbf{v}_1 + \mathbf{v}_2$. The vectors $\mathbf{v}_1$ and $\mathbf{v}_2$ are the unit-length eigenvectors of the $\mathbf{A}$ matrix which correspond to the eigenvalues $\lambda_1$ and $\lambda_2$. Write the solution $\mathbf{x}(t)=(\Delta \theta(t),\Delta q(t))^T$ in terms of $t$, $\lambda_1$, $\lambda_2$, $\mathbf{v}_1$, and $\mathbf{v}_2$. 
    \item Calculate the eigenvectors $\mathbf{v}_1$ and $\mathbf{v}_2$.
    \item Describe the behavior of $\Delta \theta$ over time using the eigenvalues you calculated in Part (1).
    % \item If you were given the values of $\lambda_1$ and $\mathbf{v}_1$ and both were complex, would you be able to write down $\lambda_2$ and $\mathbf{v}_2$ if you didn't know $k_1$ or $k_2$? Why or why not? 
\end{enumerate}
\end{question}

\vspace{0.1cm}

\clearpage

\begin{question}
    Recall the test rig described in Homework W2, Problem 3. The test rig in the diagram below is used to measure the thrust of a rotor, $T$. The rotor, which has mass $m$, is mounted at the end of a massless rod of length $l$, which has a torsional spring with stiffness $k$ at its base.
\begin{center}
\includegraphics[width=2in]{rotor-test-diagram.pdf}
\end{center}
The equation of motion is
\begin{equation}
    \ddot{\theta} = \frac{-mgl\cos(\theta) - k\theta + Tl}{ml^2} \text{.}
\end{equation}
If $k$ is known, the equilibrium thrust $T_0$ can be measured with $T_0 = mg\cos(\theta_0) + \frac{k \theta_0}{l}$.

Describe the motion of the rig according to the linear model (e.g. ``growing oscillations, exponential decay'', etc.) after a perturbation if $k > mgl\sin(\theta_0)$ and $T$ stays constant. Why does this make the rig unsuitable for its intended purpose?

\end{question}
\vspace{0.1cm}

\clearpage 

\vspace{6cm}

\begin{question}
    Suppose that the control forces and moments for a quadrotor with arm length $d = 10cm$ and rotor moment coefficient $k_m=0.003$ are given by
\begin{equation*}
    \left[ \begin{array}{c}
        Z_c \\ L_c \\ M_c \\ N_c
    \end{array} \right] = \left[ \begin{array}{c}
        -5 N \\ 0 Nm \\ 0.2 Nm \\ 0.01 Nm
    \end{array} \right] \text{.}
\end{equation*}
    If the quadrotor has the standard rotor configuration described in class, what are the thrust forces generated by each rotor?
\end{question}

\vspace{0.1cm}


\clearpage

\begin{question}
    Consider the longitudinal dynamics of the linearized quadrotor EOM:
    \begin{equation*}
        \left(\begin{array}{c}
        \Delta \dot{x}_E \\
        \Delta \dot{u} \\
        \Delta \dot{\theta} \\
        \Delta \dot{q}
        \end{array}\right)=\left(\begin{array}{c}
        \Delta u \\
        -g \Delta \theta \\
        \Delta q \\
        \frac{1}{I_u} \Delta M_c
        \end{array}\right)
    \end{equation*}
    where $\Delta M_c$ is defined in terms of $k_1$ and $k_2$ as in question 1. 

    \begin{enumerate}
        \item Suppose a closed-loop modal analysis of the system was performed and you were given only the following values:
    \begin{equation*}
        \lambda_1 = -1.5+4.2i \hspace{10pt} \lambda_2 = -0.0023+0.037i
    \end{equation*}
    \begin{equation*}
        \mathbf{v}_1 =  \left(\begin{array}{c}
        0.005+0.0021i \\
        0.075+0.0019i \\
        0.0085+0.0065i \\
        0.05 %Too challenging?
        \end{array}\right) 
        \hspace{10pt} \mathbf{v}_2 = \left(\begin{array}{c}
        0.006+0.0089i \\
        0.095+0.0030i \\
        0.0075+0.0015i \\
        0.0120+0.0030i %Too challenging?
        \end{array}\right) 
    \end{equation*}
    True or False: It is possible to determine $\lambda_3$, $\lambda_4$, $\mathbf{v}_3$, and $\mathbf{v}_4$ if $k_1$ or $k_2$ are unknown. Explain your answer.
    
    \vspace{1ex}
    \option TRUE \hspace{1cm} \option FALSE
    \vspace{1ex}
    
    \item Consider the following eigenvalues for another quadrotor system with the same dynamics described above:
    \begin{equation*}
        \lambda_1 = -1.5 \hspace{10pt} \lambda_2 = -0.0023+0.037i
    \end{equation*}
    True or False: It is possible to determine $\lambda_3$, $\lambda_4$ if $k_1$ or $k_2$ are unknown. Explain your answer.
        
    \vspace{1ex}
    \option TRUE \hspace{1cm} \option FALSE 
    \end{enumerate}
\end{question}

\vspace{0.1cm}


\end{document}
