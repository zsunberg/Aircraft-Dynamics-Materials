\documentclass{article}

\usepackage{fullpage,amsmath,amsthm,graphicx,enumitem}
\usepackage{hyperref}
\usepackage{amssymb}
\usepackage{wasysym}
\usepackage{empheq}

%For Grid, thanks internet: https://tex.stackexchange.com/questions/247541/help-with-grid-lines-in-a-pgfplot
\usepackage{tikz}
\usepackage{pgfplots}
\usetikzlibrary{calc}

\theoremstyle{definition}
\newtheorem{question}{Question}

\newcommand{\option}{{\Large$\Square$ }}

\title{ASEN 3728 Aircraft Dynamics\\Written Homework 5}

\date{Due date listed on Gradescope.}

\begin{document}

\maketitle

\begin{question}
    Consider the following longitudinal dynamics of an F-16 flying at some trim condition with airspeed $V_a = 502.0$ ft/s, where $\Delta{\bf x} = [{\Delta u}, {\Delta w}, {\Delta q}, {\Delta \theta}]^T$:
    \begin{align*}
    \Delta {\bf \dot{x}} = 
    \left[ \begin{array}{c} 
    \Delta \dot{u} \\
    \Delta \dot{w} \\
    \Delta \dot{q} \\
    \Delta \dot{\theta}
    \end{array} \right]
    &=
    \left[ \begin{array}{cccc}
    	 -0.020 &   0.016  & -0.65  & -32.17 \\
       -0.13  & -1.019   &       454.21 & 0\\
                0 &  -0.0050     &     -1.38	 & 0   \\
             0    &     0     &     1.0 & 0 \end{array} \right]
    \left[ \begin{array}{c} 
    \Delta u \\
    \Delta w \\
    \Delta q \\
    \Delta \theta
    \end{array} \right]
    \end{align*}
    \begin{enumerate}[label=(\alph*)]
    \item Determine the natural frequency and damping ratio of the short period mode and the phugoid mode.
    \item Determine the eigenvectors (mode shapes) for the short period and phugoid modes. Normalize each eigenvector so that the term corresponding to the pitch angle is 1.0.
    \item Draw the phasor plots in terms of $\hat{u}$, $\hat{w}$, $q$, and $\Delta \theta$ for each mode. Label each component
    % \item Create the augmented state space model that includes the flight path terms $\Delta x_E$ and $\Delta z_E$. Determine the eigenvalues and eigenvectors of this new matrix. What is the physical meaning of the two new modes?
    \end{enumerate}

    % \begin{figure}[!h]
    %     \centering
    %     \begin{tikzpicture}[scale=0.75]
    %         \begin{axis}[
    %             xmin=-500,xmax=500,
    %             ymin=-500,ymax=500,
    %             grid=both,
    %             grid style={line width=.1pt, draw=gray!10},
    %             major grid style={line width=.2pt,draw=gray!50},
    %             axis lines=middle,
    %             minor tick num=5,
    %             enlargelimits={abs=0.5},
    %             axis line style={latex-latex},
    %             ticklabel style={font=\tiny,fill=white},
    %             xlabel style={at={(ticklabel* cs:1)},anchor=north west},
    %             ylabel style={at={(ticklabel* cs:1)},anchor=south west}
    %         ]
    %         \node[fill=white,circle,inner sep=0pt] (O-label) at ($(O)+(-135:10pt)$) {$O$};
    %         \end{axis}
    %     \end{tikzpicture}
    %     \caption{Mode 1}
    %     \begin{tikzpicture}[scale=0.75]
    %         \begin{axis}[
    %             xmin=-500,xmax=500,
    %             ymin=-500,ymax=500,
    %             grid=both,
    %             grid style={line width=.1pt, draw=gray!10},
    %             major grid style={line width=.2pt,draw=gray!50},
    %             axis lines=middle,
    %             minor tick num=5,
    %             enlargelimits={abs=0.5},
    %             axis line style={latex-latex},
    %             ticklabel style={font=\tiny,fill=white},
    %             xlabel style={at={(ticklabel* cs:1)},anchor=north west},
    %             ylabel style={at={(ticklabel* cs:1)},anchor=south west}
    %         ]
    %         
    %         \node[fill=white,circle,inner sep=0pt] (O-label) at ($(O)+(-135:10pt)$) {$O$};
    %         \end{axis}
    %     \end{tikzpicture}\caption{Mode 2}
    %     \begin{tikzpicture}[scale=0.75]
    %         \begin{axis}[
    %             xmin=-2,xmax=2,
    %             ymin=-2,ymax=2,
    %             grid=both,
    %             grid style={line width=.1pt, draw=gray!10},
    %             major grid style={line width=.2pt,draw=gray!50},
    %             axis lines=middle,
    %             minor tick num=5,
    %             enlargelimits={abs=0.5},
    %             axis line style={latex-latex},
    %             ticklabel style={font=\tiny,fill=white},
    %             xlabel style={at={(ticklabel* cs:1)},anchor=north west},
    %             ylabel style={at={(ticklabel* cs:1)},anchor=south west}
    %         ]
    %         
    %         % \coordinate (O) at (0,0);
    %         \node[fill=white,circle,inner sep=0pt] (O-label) at ($(O)+(-135:10pt)$) {$O$};
    %         \end{axis}
    %     \end{tikzpicture}\caption{Mode 3}
    %     \begin{tikzpicture}[scale=0.75]
    %         \begin{axis}[
    %             xmin=-2,xmax=2,
    %             ymin=-2,ymax=2,
    %             grid=both,
    %             grid style={line width=.1pt, draw=gray!10},
    %             major grid style={line width=.2pt,draw=gray!50},
    %             axis lines=middle,
    %             minor tick num=5,
    %             enlargelimits={abs=0.5},
    %             axis line style={latex-latex},
    %             ticklabel style={font=\tiny,fill=white},
    %             xlabel style={at={(ticklabel* cs:1)},anchor=north west},
    %             ylabel style={at={(ticklabel* cs:1)},anchor=south west}
    %         ]
    %         
    %         % \coordinate (O) at (0,0);
    %         \node[fill=white,circle,inner sep=0pt] (O-label) at ($(O)+(-135:10pt)$) {$O$};
    %         \end{axis}
    %     \end{tikzpicture}\caption{Mode 4}
    % \end{figure}
    
\end{question}

\vspace{0.1cm}
\clearpage

\begin{question}
    The control matrix ${\bf B_{\delta_e}}$ from the elevator angle to the state derivative for the F-16 from Problem 3 is
    \begin{equation} \nonumber
    {\bf B_{\delta_e}} = 
    \left[ \begin{array}{c} 
    -.244 \\
    -1.46 \\
    -.20 \\
    0
    \end{array} \right]
    \end{equation}
    
    \begin{enumerate}[label=(\alph*)]
    \item Calculate the natural frequency and damping ratio of the short period mode approximation. To create the short period approximation use the terms in the longitudinal matrix from Problem 3.
    \item Consider the control law $\Delta \delta_e = -k_1 \Delta q - k_2 \Delta \alpha$ where $k_1 = -5.0$ and $k_2 = -0.005$. Using the short period mode approximation, calculate the new eigenvalues of the short period mode with this control law. \label{item:control}
    \item What are the eigenvalues of the full closed loop state space model using the control law from Part~\ref{item:control}? How do they compare to the approximation from Part~\ref{item:control}?
    \end{enumerate}
\end{question}

\vspace{0.1cm}
\clearpage

\vspace{6cm}

\begin{question}
    There have been several historical flight emergencies in which a multi-engine airliner has suffered a complete loss of hydraulic fluid, but the pilots have attempted to fly the aircraft using only throttle controls. In this case, the average throttle across all engines, $\delta_t$, is the relevant control input for the longitudinal dynamics, and managing the phugoid mode is an important challenge.

    The phugoid mode for a large airliner can be approximated with
    $$
    \begin{bmatrix}
        \Delta \dot{u} \\
        \Delta \dot{\theta}
    \end{bmatrix}
    = 
    \begin{bmatrix}
        -0.0025 & -30 \\
        0.0001 & 0
    \end{bmatrix}
    \begin{bmatrix}
        \Delta u \\
        \Delta \theta
    \end{bmatrix}
    +
    \begin{bmatrix}
        10 \\
        0
    \end{bmatrix}
    \Delta \delta_t \text{.}
    $$
    
    \begin{enumerate}[label=(\alph*)]
        \item If the airspeed is $u_0 = 500\text{ft/s}$, what is the natural frequency of the phugoid mode predicted by the Lanchester approximation?
        \item Why is the phugoid mode more challenging to manage than the short period mode in this situation?
        \item Find the values of the natural frequency, $\omega_n$, and damping ratio, $\zeta$, of the uncontrolled (open-loop) system expressed in the matrix equation above. How does the natural frequency compare to the Lanchester approximation?
        \item Consider the control law $\Delta \delta_t = -k_u \Delta u - k_\theta \Delta \theta$. Write down the closed-loop controlled approximation of the phugoid mode in matrix form in terms of the gains.
        \item Choose one of the control gains, $k_u$ or $k_\theta$, and propose a value that will make the closed-loop damping ratio close to 0.5 (leave the other gain at 0).
        \item How would you qualitatively describe the control law calculated above to a pilot? Indicate your answer by circling the correct italicized words in the sentence below:
    
            \center{When the (\textit{airspeed} \textbar{} \textit{pitch angle}) increases, (\textit{increase} \textbar{} \textit{reduce}) the throttle.}
    \end{enumerate}
\end{question}
\vspace{0.1cm}

\end{document}
