\begin{question}
    There have been several historical flight emergencies in which a multi-engine airliner has suffered a complete loss of hydraulic fluid, but the pilots have attempted to fly the aircraft using only throttle controls. In this case, the average throttle across all engines, $\delta_t$, is the relevant control input for the longitudinal dynamics, and managing the phugoid mode is an important challenge.

    The phugoid mode for a large airliner can be approximated with
    $$
    \begin{bmatrix}
        \Delta \dot{u} \\
        \Delta \dot{\theta}
    \end{bmatrix}
    = 
    \begin{bmatrix}
        -0.0025 & -30 \\
        0.0001 & 0
    \end{bmatrix}
    \begin{bmatrix}
        \Delta u \\
        \Delta \theta
    \end{bmatrix}
    +
    \begin{bmatrix}
        10 \\
        0
    \end{bmatrix}
    \Delta \delta_t \text{.}
    $$
    
    \begin{enumerate}[label=(\alph*)]
        \item If the airspeed is $u_0 = 500\text{ft/s}$, what is the natural frequency of the phugoid mode predicted by the Lanchester approximation?
        \item Why is the phugoid mode more challenging to manage than the short period mode in this situation?
        \item Find the values of the natural frequency, $\omega_n$, and damping ratio, $\zeta$, of the uncontrolled (open-loop) system expressed in the matrix equation above. How does the natural frequency compare to the Lanchester approximation?
        \item Consider the control law $\Delta \delta_t = -k_u \Delta u - k_\theta \Delta \theta$. Write down the closed-loop controlled approximation of the phugoid mode in matrix form in terms of the gains.
        \item Choose one of the control gains, $k_u$ or $k_\theta$, and propose a value that will make the closed-loop damping ratio close to 0.5 (leave the other gain at 0).
        \item How would you qualitatively describe the control law calculated above to a pilot? Indicate your answer by circling the correct italicized words in the sentence below:
    
            \center{When the (\textit{airspeed} \textbar{} \textit{pitch angle}) increases, (\textit{increase} \textbar{} \textit{reduce}) the throttle.}
    \end{enumerate}
\end{question}