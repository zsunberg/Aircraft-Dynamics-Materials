

\begin{question}
    Consider the following longitudinal dynamics of an F-16 flying at some trim condition with airspeed $V_a = 502.0$ ft/s, where $\Delta{\bf x} = [{\Delta u}, {\Delta w}, {\Delta q}, {\Delta \theta}]^T$:
    \begin{align*}
    \Delta {\bf \dot{x}} = 
    \left[ \begin{array}{c} 
    \Delta \dot{u} \\
    \Delta \dot{w} \\
    \Delta \dot{q} \\
    \Delta \dot{\theta}
    \end{array} \right]
    &=
    \left[ \begin{array}{cccc}
    	 -0.020 &   0.016  & -0.65  & -32.17 \\
       -0.13  & -1.019   &       454.21 & 0\\
                0 &  -0.0050     &     -1.38	 & 0   \\
             0    &     0     &     1.0 & 0 \end{array} \right]
    \left[ \begin{array}{c} 
    \Delta u \\
    \Delta w \\
    \Delta q \\
    \Delta \theta
    \end{array} \right]
    \end{align*}
    \begin{enumerate}[label=(\alph*)]
    \item Determine the natural frequency and damping ratio of the short period mode and the phugoid mode.
    \item Determine the eigenvectors (mode shapes) for the short period and phugoid modes. Normalize each eigenvector so that the term corresponding to the pitch angle is 1.0.
    \item Draw the phasor plots in terms of $\hat{u}$, $\hat{w}$, $q$, and $\Delta \theta$ for each mode. Label each component
    % \item Create the augmented state space model that includes the flight path terms $\Delta x_E$ and $\Delta z_E$. Determine the eigenvalues and eigenvectors of this new matrix. What is the physical meaning of the two new modes?
    \end{enumerate}

    % \begin{figure}[!h]
    %     \centering
    %     \begin{tikzpicture}[scale=0.75]
    %         \begin{axis}[
    %             xmin=-500,xmax=500,
    %             ymin=-500,ymax=500,
    %             grid=both,
    %             grid style={line width=.1pt, draw=gray!10},
    %             major grid style={line width=.2pt,draw=gray!50},
    %             axis lines=middle,
    %             minor tick num=5,
    %             enlargelimits={abs=0.5},
    %             axis line style={latex-latex},
    %             ticklabel style={font=\tiny,fill=white},
    %             xlabel style={at={(ticklabel* cs:1)},anchor=north west},
    %             ylabel style={at={(ticklabel* cs:1)},anchor=south west}
    %         ]
    %         \node[fill=white,circle,inner sep=0pt] (O-label) at ($(O)+(-135:10pt)$) {$O$};
    %         \end{axis}
    %     \end{tikzpicture}
    %     \caption{Mode 1}
    %     \begin{tikzpicture}[scale=0.75]
    %         \begin{axis}[
    %             xmin=-500,xmax=500,
    %             ymin=-500,ymax=500,
    %             grid=both,
    %             grid style={line width=.1pt, draw=gray!10},
    %             major grid style={line width=.2pt,draw=gray!50},
    %             axis lines=middle,
    %             minor tick num=5,
    %             enlargelimits={abs=0.5},
    %             axis line style={latex-latex},
    %             ticklabel style={font=\tiny,fill=white},
    %             xlabel style={at={(ticklabel* cs:1)},anchor=north west},
    %             ylabel style={at={(ticklabel* cs:1)},anchor=south west}
    %         ]
    %         
    %         \node[fill=white,circle,inner sep=0pt] (O-label) at ($(O)+(-135:10pt)$) {$O$};
    %         \end{axis}
    %     \end{tikzpicture}\caption{Mode 2}
    %     \begin{tikzpicture}[scale=0.75]
    %         \begin{axis}[
    %             xmin=-2,xmax=2,
    %             ymin=-2,ymax=2,
    %             grid=both,
    %             grid style={line width=.1pt, draw=gray!10},
    %             major grid style={line width=.2pt,draw=gray!50},
    %             axis lines=middle,
    %             minor tick num=5,
    %             enlargelimits={abs=0.5},
    %             axis line style={latex-latex},
    %             ticklabel style={font=\tiny,fill=white},
    %             xlabel style={at={(ticklabel* cs:1)},anchor=north west},
    %             ylabel style={at={(ticklabel* cs:1)},anchor=south west}
    %         ]
    %         
    %         % \coordinate (O) at (0,0);
    %         \node[fill=white,circle,inner sep=0pt] (O-label) at ($(O)+(-135:10pt)$) {$O$};
    %         \end{axis}
    %     \end{tikzpicture}\caption{Mode 3}
    %     \begin{tikzpicture}[scale=0.75]
    %         \begin{axis}[
    %             xmin=-2,xmax=2,
    %             ymin=-2,ymax=2,
    %             grid=both,
    %             grid style={line width=.1pt, draw=gray!10},
    %             major grid style={line width=.2pt,draw=gray!50},
    %             axis lines=middle,
    %             minor tick num=5,
    %             enlargelimits={abs=0.5},
    %             axis line style={latex-latex},
    %             ticklabel style={font=\tiny,fill=white},
    %             xlabel style={at={(ticklabel* cs:1)},anchor=north west},
    %             ylabel style={at={(ticklabel* cs:1)},anchor=south west}
    %         ]
    %         
    %         % \coordinate (O) at (0,0);
    %         \node[fill=white,circle,inner sep=0pt] (O-label) at ($(O)+(-135:10pt)$) {$O$};
    %         \end{axis}
    %     \end{tikzpicture}\caption{Mode 4}
    % \end{figure}
    
\end{question}